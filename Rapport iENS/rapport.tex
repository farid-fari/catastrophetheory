\documentclass{article}
\usepackage[T1]{fontenc}
\usepackage[utf8]{inputenc}
\usepackage[french]{babel}
\usepackage{cite}
\usepackage{url}
\usepackage{hyperref}
\usepackage{amsfonts}
\usepackage{amsmath}
\usepackage{amsthm}

\renewcommand{\thesection}{\Roman{section}}
\renewcommand{\thesubsection}{\thesection.\arabic{subsection}}

\newcommand{\cont}{\mathcal{C}^0}
\newcommand{\cun}{\mathcal{C}^1}
\newcommand{\cinf}{\mathcal{C}^\infty}
\newcommand{\R}{\mathbb{R}}

\newtheorem{defn}{Définition}
\newtheorem{thm}{Théorème}
\newtheorem{lemm}{Lemme}

\begin{document}

\title{La théorie des catastrophes}
\author{Farid Arthaud}

\maketitle
\tableofcontents

\section{Introduction}

La théorie des catastrophes est l'étude de variations soudaines -- ou discontinuités -- naissant de perturbations continues.
Elle découle intuitivement de questionnements physiques, notamment du comportement de certains jouets \cite{bbcvid} mais aussi de problèmes concrets comme le roulis d'un navire \cite{poston} ou le flambage d'une poutre.

La théorie des catastrophes est née dans les années 60, quasiment entièrement du travail de \textbf{René Thom} puis \textbf{Christopher Zeeman}.
Dans son livre \textit{Stabilité structurelle et morphogenèse} \cite{thom}, le théorème de classification de \textsc{Thom} est établi: sous certaines hypothèses (telles que la \textit{stabilité structurelle}), les singularités des familles d'au plus cinq applications peuvent être classifiées en \textbf{sept catégories}.
Il s'agit du pli, de la fronce, la queue d'aronde, la vague, le poil, le papillon et le champignon \cite{wikipedia}.
Malgré les noms fantaisistes, il s'agit de catégories précises qui comprennent l'arité des fonctions et leur forme qualitative au voisinage de la singularité.

Une des applications principales de cette théorie sont les simulations: de phénomènes physiques mais aussi sociaux (comportements durant des révoltes, ...) et économiques \cite{bbcvid} \cite{poston}.
Son utilité pour la physique découle de ce que \textbf{Demazure} appelle `la philosophie de la théorie des catastrophes' \cite{demazure}: étudier ces variétés $\cinf$ peut être vu comme l'étude d'une fonction potentiel définie sur l'espace des états possibles d'un système physique.
La théorie des catastrophes s'intéresse en fait aux états d'équilibre de solutions d'équations différentielles en fonction de la variation des paramètres.

Alors, l'étude d'une singularité devient celle d'un équilibre, et leur classification est précieuse pour le physicien.
Cependant, nombre de ses applications sont controversées, notamment celles qui essaient d'expliquer un comportement humain par des mathématiques, pour des raisons éthiques \cite{wikipedia}.

\section{Catastrophes}

\subsection{Points critiques, variétés}

\begin{defn}
	point critique, dégénrescence
Un point critique $a$ est dit non dégénéré si $\forall x, rg(d^2f_a(x))=n$ (c'est à dire que $x\mapsto(y\mapsto d^2f_a(x,y))$ est isomorphisme), ou de manière équivalente si le déterminant Hessien de $f$ est non nul.
\end{defn}
\begin{defn}
	immersion submersion
\end{defn}
\begin{defn}
	plongement
\end{defn}

\begin{defn}
	Une \textbf{variété} est un ensemble de $\R^n$ qui est localement homéomorphe à $\R^m$ et qui a un unique plan tangent en tout point, où $m\leq n$ est appelé la \textit{dimension de la variété}.
	Un homéomorphisme de  la variété vers $\R^m$ est appelé un \textit{système de coordonées locales}.
\end{defn}

\subsection{Transversalité et stabilité structurelle}

\begin{defn}
Deux sous-espaces vectoriels de $\R^n$ $F$ et $G$ sont dit \textbf{transverses} s'ils s'intersectent en un espace de dimension minimale, ou de manière équivalente si $F+G=\R^n$.

On étend la définition aux familles finies par $codim(T_1\cap...\cap T_m)=codim(T_1)+...+codim(T_m)$.
\end{defn}

Une première condition de transversalité est: $U$ et $V$ sont transverses si et seulement si $dim(U\cap V)=dim(U)+dim(V)-n$.
On constate que si la somme des dimensions est inférieure à $n$, les espaces ne peuvent être transverses: la transversalité dépend fortement de l'espace dans lequel on travaille.

Une autre condition de transversalité, cette fois-ci géométrique, est que pour toute famille d'espaces affines respectivement dirigés selon $T_1,...,T_m$, leur intersection ne doit pas être vide.

On généralise ensuite la définition à deux variétés: elles sont transverses si leurs plans tangents sont transverses en tout point de leur intersection.
Puisque les espaces tangents de la variété sont de même dimension que la variété, la somme des dimensions des variétés doit être au moins $n$ pour qu'elles soient transverses (si elles s'intersectent).

On peut donc redéfinir une variété comme l'intersection locale transverse d'hypersurfaces, qui correspondent aux noyaux des coordonnées locales.

Une application est transverse à une hypersurface ssi ...

\begin{defn}
Une application $f$ est dit \textbf{structurellement stable} si pour toute application $p$ suffisament proche de 0 (et infiniment continue), $f+p$ a le même type de point critique que $f$ à translation près.
\end{defn}

Un point critique est en fait structurellement stable si et seulement si il est non dégénéré.
La \textbf{stabilité structurelle} signifie plus généralement que le comportement qualitatif ne change pas malgré une perturbation suffisament petite.

Le \textbf{théorème d'isotopie de \textsc{Thom}} établit un premier lien entre ces deux notions-clés:

\begin{thm}
Les intersections transverses sont structurellement stables.
\end{thm}

Le \textbf{théorème de transversalité de \textsc{Thom}} nous informe enfin sur la généricité des applications transverses à une variété donnée.
La transversalité est donc un `positionnement générique', dans le sens de la densité.

\begin{thm}
Une application peut être déformée de manière arbitrairement fine en une application transverse à toute variété donnée.
\end{thm}

La version faible du théorème de transversalité s'énonce à partir d'une fonction $f: V\times\Lambda\to F$, où $\Lambda\in G$ est un ouvert servant à `indicer' une famille de fonctions.

\begin{thm}
	Si $f$ est transverse à $W$, une sous-variété de $F$, alors l'ensemble des $\lambda\in\Lambda$ tels que $x\mapsto f(x,\lambda)$ soit transverse à $W$ est dense de $\Lambda$ en tant qu'intersection dénombrable d'ouverts dense.
\end{thm}

\subsection{Réecriture au voisinage d'un point critique}

La classification des points critiques demande une forme d'`équivalence' permettant de les mettre sous une forme canonique les rendant comparables.
On a pour cela à notre disposition un certain nombre de lemmes et théorèmes permettant la réecriture d'une fonction au voisinage d'un point critique.

Tout d'abord vient le fameux \textbf{théorème d'inversion locale}, qui permet de ...

\begin{thm}
Si $f\in\cinf(U,\R^m)$, et que $df_a$ est bijective (ou de déterminant non nul), alors f est un \textbf{difféomorphisme local} (ou localement bijective de réciproque $\cinf$).
\end{thm}

\textit{Preuve}: voir annexe

Ceci indique en particulier que si $f:\R^m\mapsto\R^n$ est de rang $m$ en un de ses zéros, alors son noyau peut être paramétré de manière continue.
La réciproque est en fait vraie: toute variété suffisament continue dans $\R^n$ de dimension $m$ peut localement être vue comme $f^{-1}(0)$ avec $f:\R^m\mapsto\R^n$.

Ceci crée donc une équivalence entre les variétés de dimension $(n-m)$ et les ensembles définis par $m$ équations (le noyau de chaque composante de $f$).

Le \textbf{lemme d'\textsc{Hadamard}} permet de faire intervenir les coordonnées de $f$ dans son écriture au voisinage de $0$, nous permettant de nous rapprocher d'une écriture standard.

\begin{lemm}
Si $f(0)=0$ et $f$ est infiniment différentiable, alors il existe un voisinage de 0 sur lequel $f$ peut s'écrire: $f(x)=g_1(x)x_1+...+g_n(x)x_n$ avec $g_i(0) = \frac{\partial f}{\partial x_i}(0)$.

Si 0 est point critique, on peut réitérer cela et écrire (toujours sur un voisinage de 0): $$f(x)=\sum_{i,j} x_ix_jh_{ij}(x)$$
\end{lemm}

Ce dernier lemme permet de démontrer le \textbf{lemme de \textsc{Morse}}, qui donne une expression nettement plus simple au voisinage de $0$.

\begin{lemm}
Si $f\in\cinf(\R^n,\R)$ admet un point critique non dégénéré en $u$, alors il existe un voisinage de $u$ et un $\cinf$-difféomorphisme (donc un changement de coordonnées) $y=(y_1,...,y_n)$ tel que: $$f=f(u)+y_1^2+...+y_l^2-y_{l+1}^2-...-y_n^2$$
\end{lemm}

\textit{Preuve}: voir annexe

\begin{defn}
Une application du type $a+y_1^2+...+y_k^2-y_{k+1}^2-...-y_n^2$ est appelée \textit{k-saddle de \textsc{Morse}}.
\end{defn}

De plus, on note que si $f$ est \textit{0-saddle} alors $u$ est minimum local, et si $f$ est \textit{n-saddle}, alors $u$ est maximum local.
$u$ est alors un point critique isolé (il existe un voisinage sans autre points critiques), et puisque le changement de coordonées conserve l'isolation, on en déduit que tout point critique non dégénéré est isolé. De plus, $l$ est invariant par changement de coordonnées (donc par difféomorphisme, c'est une propriété topologique intrinsèque à $f$).

Dans le cas $n=1$, si $f(0)=f'(0)=...=f^{(k-1)}=0$ et $f^{(k)}(0)\neq 0$ alors, il existe un voisinage de 0 et un changement de coordonées pour lesquels $f(x)=\pm x^k$ (qui est un $+$ si $k$ est impair).

On définit la relation d'équivalence \textit{en 0} sur $\cinf$ où $f \mathcal{R} g$ si et seulement si $g=f\circ y + \gamma$ sur un voisinage de 0, avec $y$ $\cinf$-difféomorphisme et $\gamma$ une constante pour pallier aux éventuelles différences de valeur en 0.

Deux fonctions sont en relation si et seulement si leurs écritures en ces coordonnées $\pm x^k$ et $\pm x^l$ ont le même signes et $k=l$. Ceci fournit une classification des points critiques en dimension 1, en fonction de leurs dérivées successives.

Vient enfin le \textbf{théorème de décomposition} (traduit de \textit{splitting lemma}), dont l'énoncé suit.

\begin{thm}
Si f admet 0 pour point critique et $H(f)_0$ est de rang $r$, alors $f$ est équivalente (au sens défini ci-dessus) au voisinage de 0 à une fonction de la forme $$g(x)=\pm x_1^2\pm x_2^2\pm ... \pm x_r^2 + h(x_{r+1},...,x_n)$$
\end{thm}

\textit{Preuve (en dimension deux)}: voir annexe


\subsection{Germes, jets}

On introduit la notion de \textit{k-jet}, la fonction polynomiale formée par les $k$ premiers termes du développement de Taylor d'une fonction en un point $y$, noté: $$(j^k_y f)(y+x) = f(y) + xf'(y)+...+\frac{x^n f^{(n)}(y)}{n!}$$

Pour une fonction à $n$ variables, on définit son dévelopmmeent de Taylor et son \textit{k-jet} par: $$(j^k_y f)(y+x) = f(y) + df_y(x)+...+\frac{(df^{(k)}_y)(x)(x)...(x)}{k!}$$

On peut réecrire le terme d'ordre $k$ sous la forme $$\frac{1}{k!}\sum_{i_1,...,i_k} \frac{\partial^k f}{\partial x_{i_1}...x_{i_k}} x_{i_1}...x_{i_k}$$

\section{Classification des catastrophes}
\subsection{Théorèmes de Sard et de Whitney}
Ces deux théorèmes, dont les énoncés généraux sont admis, permettent de ...

\begin{thm}
	Si $V$ est une variété compacte de dimension $n$, il existe un plongement de $V$ dans $\R^{2n+1}$.
\end{thm}

\begin{thm}
	Si $f: U\to F$ où $U$ est un ouvert de $E$ est de classe $\cinf$ alors l'ensemble des images de ses points critiques est négligeable dans $F$.
\end{thm}

\begin{proof}
	\textit{Voir annexe pour $E=F=\R$}.
\end{proof}

Sous sa forme géométrique, ce théorème dit que

\subsection{Le pli et la fronce}
\subsection{Les autres catastrophes}

\section{Bibliographie}

\renewcommand\refname{\vskip -1cm}
\bibliography{sources}
\bibliographystyle{unsrt}

\end{document}
