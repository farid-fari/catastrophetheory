\documentclass{article}
\setlength{\parindent}{0pt}
\usepackage[utf8]{inputenc}
\usepackage[french]{babel}
\usepackage{amsfonts}
\usepackage{cite}
\usepackage{url}

\begin{document}

\title{Théorie des catastrophes: singularités des applications différentiables}
\author{Farid Arthaud}
\maketitle

\section{Positionnements thématiques et mots-clés}

\textbf{Thème}: Analyse

\textbf{Mots-clés}:
\begin{enumerate}
    \item Singularités
    \item Théorie des catastrophes
    \item Calcul différentiel
    \item Stabilité
    \item Simulations
\end{enumerate}

\textbf{Keywords}:

\begin{enumerate}
    \item Catastrophe theory
    \item Singularities
    \item Smoothness
    \item Stability
    \item Simulations
\end{enumerate}

\section{Bibliographie commentée}

La théorie des catastrophes est l'étude de variations soudaines -- ou discontinuités -- naissant de perturbations continues.
Elle découle intuitivement de questionnements physiques, notamment du comportement de certains jouets \cite{bbcvid} mais aussi de problèmes concrets comme le roulis d'un navire \cite{poston} ou le flambage d'une poutre.

La théorie des catastrophes est née dans les années 60, quasiment entièrement du travail de \textbf{René Thom} puis \textbf{Christopher Zeeman}.
Dans son livre \textit{Stabilité structurelle et morphogenèse} \cite{thom}, le théorème de classification de \textsc{Thom} est établi: sous certaines hypothèses (telles que la \textit{stabilité structurelle}), les singularités des familles d'au plus cinq applications peuvent être classifiées en \textbf{sept catégories}.
Il s'agit du pli, de la fronce, la queue d'aronde, la vague, le poil, le papillon et le champignon \cite{wikipedia}.
Malgré les noms fantaisistes, il s'agit de catégories précises qui comprennent l'arité des fonctions et leur forme qualitative au voisinage de la singularité.

Une des applications principales de cette théorie sont les simulations: de phénomènes physiques mais aussi sociaux (comportements durant des révoltes, ...) et économiques \cite{bbcvid} \cite{poston}.
Son utilité pour la physique découle de ce que \textbf{Demazure} appelle `la philosophie de la théorie des catastrophes' \cite{demazure}: étudier ces variétés $\mathcal{C}^\infty$ peut être vu comme l'étude d'une fonction potentiel définie sur l'espace des états possibles d'un système physique.
La théorie des catastrophes s'intéresse en fait aux états d'équilibre de solutions d'équations différentielles en fonction de la variation des paramètres.

Alors, l'étude d'une singularité devient celle d'un équilibre, et leur classification est précieuse pour le physicien.
Cependant, nombre de ses applications sont controversées, notamment celles qui essaient d'expliquer un comportement humain par des mathématiques, pour des raisons éthiques \cite{wikipedia}.

\bigskip

\small{263 mots / 650 max}

\section{Problématique retenue}

Comment un problème si large se découpe en si peu de cas, et quel sont le sens et le comportement de ces catégories établies par \textbf{Thom}?

\section{Objectifs du TIPE}

Saisir les grandes lignes de l classification des singularités des applications différentiables, et comprendre intuitivement et formellement ce théorème et ses conséquences.

\section{Liste des références bibliographiques}

\renewcommand\refname{\vskip -1cm}
\bibliography{sources}
\bibliographystyle{unsrt}

\end{document}
