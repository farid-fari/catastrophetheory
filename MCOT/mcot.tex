\documentclass{article}
\setlength{\parindent}{0pt}
\usepackage[utf8]{inputenc}
\usepackage[french]{babel}
\usepackage{amsfonts}
\usepackage{cite}

\begin{document}

\title{Singularités des applications différentiables}
\author{Farid Arthaud}
\maketitle

\section{Positionnements thématiques et mots-clés}

\textbf{Thème}: Analyse

\textbf{Mots-clés}:
\begin{enumerate}
    \item Singularités
    \item Théorie des catastrophes
    \item Calcul différentiel
    \item Stabilité
    \item Variations
\end{enumerate}

\textbf{Keywords}:

\begin{enumerate}
    \item Catastrophe theory
    \item Singularities
    \item Smoothness
    \item Stability
    \item Variations
\end{enumerate}

\section{Bibliographie commentée}

\textbf{Thom} commence son ouvrage \textit{Stabilité structurelle et morphogenèse} par une citation sur la forme des vagues et des dunes, avant de questionner le sens de la vie \cite{thom}.
Ceci illustre l'importance que la théorie des catastrophes a eu lors de sa genèse.

Elle découle intuitivement de questionnements physiques, notamment du comportement de certains jouets \cite{bbcvid} mais aussi de problèmes concrects comme le roulis d'un navire \cite{Poston}.

La théorie des catastrophes est née dans les années 70, quasiment entièrement du travail de \textbf{René Thom}.
Dans son livre \textit{Stabilité structurelle et morphogenèse} \cite{thom},

\section{Problématique retenue}

Comment un problème si large se résume en si peu, et quel sont le sens et le comportement de ces catégories établies par \textbf{Thom}?

\section{Objectifs du TIPE}

Saisir les grandes lignes de l classification des singularités des applications différentiables, et comprendre intuitivement et formellement ce théorème et ses conséquences.

\section{Liste des références bibliographiques}

\bibliography{sources}
\bibliographystyle{acm}

\end{document}
