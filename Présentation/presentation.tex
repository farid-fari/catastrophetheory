\documentclass{beamer}
\usepackage[utf8]{inputenc}
\usepackage[T1]{fontenc}
\usepackage{lmodern}
\usepackage[french]{babel}
%\usepackage{amsfonts}
%\usepackage{amsmath}
%\usepackage{amsthm}

\useinnertheme{default}
\useoutertheme{miniframes}
\usecolortheme{seahorse}

\author{Farid Arthaud}
\title{Théorie des Catastrophes}
\date{\today}

\begin{document}
\frame{\titlepage}

\frame{\tableofcontents}

\section*{Introduction}
\begin{frame}{Exemple: le baigneur}
    \begin{itemize}
        \item Contrôle absolu sur la précision
        \item `Virage' soudain et incontournable lors du réglage du paramètre
        \pause
        \item \textbf{Explication}: disparition d'un équilibre d'un système caché
    \end{itemize}
\end{frame}
\begin{frame}{Contexte historique}
    \begin{itemize}[<+->]
        \item René \textsc{Thom}: \textit{Stabilité structurelle et morphogenèse} (1972)
        \item Cristopher \textsc{Zeeman}:
        \item Popularisation (années 70) et applications diverses (biologie, sociologie, ...)
        \item Controverse, remise en question de la pertinence des applications
    \end{itemize}
    \begin{quote}<4>
        Les choses qui changent soudainement, par à-coups, ont longtemps résisté à toute analyse mathématique.
        Une méthode dérivée de la topologie décrit ces phénomènes comme des exemples de sept `catastrophes élémentaires'.
    \end{quote}
\end{frame}

\section{Catastrophes}
\subsection{Variétés et points critiques}
\subsection{Transversalité et stabilité structurelle}
\subsection{Réecriture au voisinage d'un point critique}
\subsection{Germes et jets}

\section{Théorème de classification}
\subsection{Le théorème de Whitney}
\subsection{Le pli et la fronce}
\subsection{Les autres catastrophes}

\section{Applications physiques}
\subsection{Equilibre thermodynamique}
\subsection{Roulis d'un navire}

\section*{Conclusion}
\frame{}

\end{document}
