\documentclass{beamer}
\usepackage[utf8]{inputenc}
\usepackage[T1]{fontenc}
\usepackage{lmodern}
\usepackage[french]{babel}
\usepackage{amsfonts}
\usepackage{amsmath}
\usepackage{amsthm}

\useinnertheme{default}
\useoutertheme{miniframes}
\usecolortheme{seahorse}

\newcommand{\cont}{\mathcal{C}^0}
\newcommand{\cun}{\mathcal{C}^1}
\newcommand{\cinf}{\mathcal{C}^\infty}
\newcommand{\R}{\mathbb{R}}
\newcommand{\N}{\mathbb{N}}

\newtheorem{thm}{Théorème}
\newtheorem{lemm}{Lemme}
\theoremstyle{definition}
\newtheorem{defn}{Définition}

\author{Farid Arthaud}
\title{Théorie des Catastrophes}
\date{\today}

\begin{document}
\frame{\titlepage}

\frame{\tableofcontents}

\section*{Introduction}
\begin{frame}{Exemple: le baigneur}
    \begin{itemize}
        \item Contrôle absolu sur la précision
        \item `Virage' soudain et incontournable lors du réglage du paramètre
        \pause
    \end{itemize}
    \textbf{Explication}: disparition d'un équilibre d'un système caché
\end{frame}

\begin{frame}{Contexte historique}
    \begin{itemize}
        \item René \textsc{Thom}: \textit{Stabilité structurelle et morphogenèse} (1972)
        \item Cristopher \textsc{Zeeman}: Recherche intensive aux cotés de \textsc{Thom}
        \item Popularisation (années 70) et applications diverses (biologie, sociologie, ...)
        \item Controverse, remise en question de la pertinence des applications
    \end{itemize}
    \begin{quote}<2>
        ``Les choses qui changent soudainement, par à-coups, ont longtemps résisté à toute analyse mathématique.
        Une méthode dérivée de la topologie décrit ces phénomènes comme des exemples de sept `catastrophes élémentaires'.'' - C. Zeeman
    \end{quote}
\end{frame}

\section{Catastrophes}
\subsection{Variétés et points critiques}
\begin{frame}{Points critiques}
    $f: E \to F, \cinf$
    \begin{defn}
        Un point critique pour $f$ est un point $a$ tel que $df_a$ ne soit pas surjective.

	    Un point critique $a$ est dit non dégénéré si $\forall x, d^2f_a(x)$ est un isomorphisme, ou de manière équivalente si le déterminant Hessien de $f$ est non nul. (??)
    \end{defn}
\end{frame}

\begin{frame}{Sous-variétés}
    \begin{defn}
        Une \textbf{sous-variété} $V$ de $\R^n$ est un ensemble qui est localement descriptible par un \textbf{système non dégénéré d'équations locales},il existe pour tout point $a\in V$, $\Phi_1,...,\Phi_m$ telles que:
        \begin{enumerate}[<+->]
            \item pour un voisinage $U$ de $a$, $U\cap\ker(\Phi_1,...,\Phi_m)=U\cap V$
            \item $d\Phi_1,...,d\Phi_m$ forment un système libre de formes linéaires.
        \end{enumerate}

        $m$ s'appelle la codimension de la sous-variété, que l'on suppose être la même en tout point de $V$ par la suite (on peut supposer la variété connexe par exemple).
    \end{defn}
    \pause
    \begin{thm}
        Une variété de dimension $p$ admet en tout point un espace tangent de dimension $p$.
        C'est en fait l'intersection des noyaux du système d'équations locales définissant $V$ en ce point.
    \end{thm}
\end{frame}

\subsection{Transversalité et stabilité structurelle}
\subsection{Réecriture au voisinage d'un point critique}
\subsection{Germes et jets}

\section{Théorème de classification}
\subsection{Le théorème de Whitney}
\subsection{Le pli et la fronce}
\subsection{Les autres catastrophes}

\section{Applications physiques}
\subsection{Equilibre thermodynamique}
\subsection{Roulis d'un navire}

\section*{Conclusion}
\frame{}

\end{document}
