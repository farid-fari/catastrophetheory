\documentclass[compress]{beamer}
\usepackage[utf8]{inputenc}
\usepackage[T1]{fontenc}
\usepackage{lmodern}
\usepackage[french]{babel}
\usepackage{amsfonts}
\usepackage{amsmath}
\usepackage{amsthm}

\useinnertheme{default}
\useoutertheme{miniframes}
\usecolortheme{beaver}
\setbeamercovered{transparent}

\newcommand{\cont}{\mathcal{C}^0}
\newcommand{\cun}{\mathcal{C}^1}
\newcommand{\cinf}{\mathcal{C}^\infty}
\newcommand{\R}{\mathbb{R}}
\newcommand{\N}{\mathbb{N}}

\newtheorem{thm}{Théorème}
\newtheorem{lemm}{Lemme}
\theoremstyle{definition}
\newtheorem{defn}{Définition}

\author{Farid Arthaud}
\title{Théorie des Catastrophes}
\date{\today}

\begin{document}
\frame{\titlepage}

\frame{\tableofcontents}

\section*{Introduction}
\begin{frame}{Exemple: le baigneur}
    \begin{itemize}[<+->]
        \item Contrôle absolu sur la précision
        \item `Virage' soudain et incontournable lors du réglage du paramètre
    \end{itemize}
    \begin{description}[<+->]
        \item[Explication:] Disparition d'un équilibre d'un système physique caché, catastrophe de type `pli'
        \item[Objectif:] Saisir la classification des catastrophes établie par \textsc{Thom} en faibles dimensions, et ses applications
    \end{description}
\end{frame}

\begin{frame}{Contexte historique}
    \begin{itemize}
        \item René \textsc{Thom}: \textit{Stabilité structurelle et morphogenèse} (1972)
        \item Cristopher \textsc{Zeeman}: Recherche intensive aux cotés de \textsc{Thom}
        \item Popularisation (années 70) et applications diverses (biologie, sociologie, ...)
        \item Controverse, remise en question de la pertinence des applications
    \end{itemize}
    \begin{quote}<2>
        ``Les choses qui changent soudainement, par à-coups, ont longtemps résisté à toute analyse mathématique.
        Une méthode dérivée de la topologie décrit ces phénomènes comme des exemples de sept `catastrophes élémentaires'.'' - C. Zeeman
    \end{quote}
\end{frame}

\section{Catastrophes}
\subsection{Variétés et points critiques}
\begin{frame}{Points critiques}
    Soit $f: E \to F, \cinf$
    \begin{defn}
        Un point critique pour $f$ est un point $a$ tel que $df_a$ ne soit pas surjective.

	    Un point critique $a$ est dit non dégénéré si $\forall x, d^2f_a(x)$ est un isomorphisme, ou de manière équivalente si le déterminant Hessien de $f$ est non nul.
    \end{defn}
    \pause
    En \textbf{dimension 1}: $a$ point critique $\iff f'(a) = 0$

    $a$ point critique dégénéré $\iff f'(a) =f''(a) = 0$
\end{frame}

\begin{frame}{Sous-variétés}
    \begin{defn}
        Une \textbf{sous-variété} $V$ de $\R^n$ est un ensemble qui est localement descriptible par un \textbf{système non dégénéré d'équations locales}, soit $m$ équations indépendantes.
        \begin{enumerate}[<+->]
            \item pour un voisinage $U$ de $a$, $U\cap\ker(\Phi_1,...,\Phi_m)=U\cap V$
            \item $d\Phi_1,...,d\Phi_m$ forment un système libre de formes linéaires.
        \end{enumerate}
        \pause[3]
        $m$ s'appelle la codimension de la sous-variété, que l'on suppose être la même en tout point de $V$ par la suite (on peut supposer la variété connexe par exemple).
    \end{defn}
    \pause[4]
    \begin{thm}
        Une variété de dimension $p$ admet en tout point un espace tangent de dimension $p$.

        C'est en fait l'intersection des noyaux du système d'équations locales définissant $V$ en ce point.
    \end{thm}
\end{frame}

\subsection{Transversalité et stabilité structurelle}
\begin{frame}{Intersections transverses}
    \begin{defn}
        Deux sous-espaces $F$ et $G$ sont dits d'intersection transverse si $F+G=E$.
    \end{defn}
    \pause
    Caractérisations:
    \begin{enumerate}
        \item<2-5> Intersection minimale
        \item<2-5> Sous-espaces affines parallèles (géométrique) s'intersectent
        \item<3-5> $dim(F\cap G)=dim(F)+dim(G)-dim(E)$
        \item<3-5> $codim(F\cap G)=codim(F)+codim(G)$
    \end{enumerate}
    \pause[4]
    Pour une famille, $codim(T_1\cap...\cap T_m)=codim(T_1)+...+codim(T_m)$
    \pause[5]
    Pour des variétés, espaces tangents transverses en tout point de l'intersection
\end{frame}

\begin{frame}{Théorèmes de transversalité}
    \begin{thm}
        Une intersection transverse de sous-variétés est une sous-variété.
    \end{thm}
    \pause
    Soit $g:V\to W, \cinf$
    \begin{thm}
        Si $a\in g^{-1}(W)$ et $g$ transverse à $W$ en $a$, alors $g^{-1}(W)$ est sous-variété en $a$ de dimension $dim_a(V)-codim_{g(a)}(W)$.
    \end{thm}
\end{frame}

\subsection{Germes et jets}
\begin{frame}{Espace des germes}
    \begin{defn}
        L'espaces des germes s'identifie à un point et aux premiers termes du développement de Taylor d'une fonction.

        $$(j^k_x f)(h) = f(x) + df_x(h)+...+\frac{df^k_x(h)(h)...(h)}{k!}$$
    \end{defn}
    \pause
    Permet d'exprimer des équations différentielles comme des variétés:

    $$f'(x)=f(x) \iff (x,j^1_xf) \in \{(a,b,c) \mid b=c \} \subseteq J^1(\R,\R)$$
\end{frame}

\section{Théorème de classification}
\subsection{Théorème de transversalité \textsc{Thom}}
\begin{frame}{Le théorème de transversalité de \textsc{Thom}}
    \begin{thm}{(de transversalité de \textsc{Thom})}
        Pour une sous-variété de l'espace des germes donnée, une fonction choisie ``au hasard'' (par densité) lui est transverse.
    \end{thm}
    \pause
    \textbf{Exemple}: $\{(a,b,c) \mid b=c \}$ est une sous-variété de dimension $2$ de $J^1(\R,\R) \simeq \R^3$ décrivant $f'=f$

    Espace tangent en $(a,b,b)$: $\left((1,0,0), (0,1,1)\right)$

    Espace tangent de $j^1f$ en $j^1_xf$: $(1,f'(x),f''(x))\notin\left((1,0,0), (0,1,1)\right)$

    Une fonction $f$ choisie ``au hasard'' en chaque point soit: ne vérifie pas $f'(x)=f(x)$, soit la vérifie en des points isolés.
\end{frame}

\subsection{Le pli et la fronce}
\begin{frame}{Contour d'une surface}
\end{frame}
\begin{frame}{Théorème de \textsc{Whitney}}
\end{frame}

\subsection{Les autres catastrophes}

\section{Applications physiques}
\subsection{La machine de Zeeman}
\frame{Expérience}
\frame{Analyse théorique}

\end{document}

\subsection{Réecriture au voisinage d'un point critique}
\begin{frame}{Lemmes d'\textsc{Hadamard}, de \textsc{Morse}, inversion locale}
    \begin{thm}{(inversion locale)}
        Si $a$ n'est pas un point critique, il existe des coordonnées locales telles que $f(x)=f(a)+x$.
    \end{thm}
    \begin{lemm}{(de \textsc{Hadamard})}
        Il existe un voisinage de $0$ sur lequel $f$ peut s'écrire: $f(x)=x_1g_1(x)+...+x_ng_n(x)$ avec $g_i(0) = \frac{\partial f}{\partial x_i}(0)$.

        Si $df_0=0$, on peut réitérer cela et écrire (toujours sur un voisinage de $0$):

        $$f(x)=\sum_{1\leq i,j \leq n} x_ix_jh_{ij}(x)$$
    \end{lemm}
    \begin{lemm}{(de \textsc{Morse})}
        Si $f$ admet un point critique non dégénéré en $u$, alors il existe un voisinage de $u$ et des coordonnées centrées en $u$ $y=(y_1,...,y_n)$ tels que:

        $$f=f(u)+y_1^2+...+y_l^2-y_{l+1}^2-...-y_n^2$$

        $l$ ne dépend pas des coordonnées choisies.
    \end{lemm}
\end{frame}
